% From mitthesis package
% Version: 1.01, 2023/06/19
% Documentation: https://ctan.org/pkg/mitthesis
%
% The abstract environment creates all the required headers and footnote. 
% You only need to add the text of the abstract itself.
%
% Approximately 500 words or less; try not to use formulas or special characters
% If you don't want an initial indentation, do \noindent at the start of the abstract

\noindent
Accurate understanding and querying of historical legal property documents remains a significant challenge for urban planning research. These records typically exist only in analog format—scanned images with inconsistent quality, archaic language, and no structured metadata. This limitation severely hinders systematic analysis of how discriminatory housing practices, particularly racial covenants, shaped city development patterns. While researchers have begun applying Generative AI systems to assist with legal documentation work,\footnote{Stanford Digital Humanities, \textit{STARA: Stanford Text Analysis for Racial Agreements}, 2024. Available at \url{https://dho.stanford.edu/wp-content/uploads/STARA.pdf}.} fundamental challenges persist in retrieval accuracy, model hallucination, and reliable extraction of structured facts from unstructured historical text.

Using racial covenant analysis as our test case, this thesis addresses two interconnected challenges. \textbf{Information Extraction}: How can we accurately extract structured spatio-temporal information from degraded historical documents? \textbf{Reasoning at Scale}: How can retrieval systems maintain accuracy when answering complex queries requiring multi-hop reasoning across thousands of documents with temporal and spatial constraints?

To address the first challenge, we developed a document processing pipeline and applied it to 569 historical deed records from a test subset of the Massachusetts Covenants Project, spanning \hl{1861--1930} in North Middlesex County. Our pipeline integrates optical character recognition, named entity extraction, and geographic information systems to transform scanned deeds into structured spatio-temporal data. The case study demonstrates successful extraction of policy-relevant keywords and accurate geolocation of \hl{$76.3\%$} of documents containing geographic information, validating the effectiveness of this approach for systematically analyzing discriminatory housing patterns. The pipeline is released as an open-source tool\footnote{Available at \url{https://github.com/lauyihong/deeds_pipeline}.} and is currently being packaged for deployment by MassHousing, a state affordable housing agency, to support ongoing collaborative research.

While the pipeline produces structured data, querying this information at scale presents its own challenges. To address this second problem, we adapt and optimize a Graph RAG (Graph-based Retrieval-Augmented Generation) framework, designing its knowledge graph structure to specifically encode the spatio-temporal relationships inherent in historical deeds. Through controlled experiments on 2,000 synthetic documents mirroring real deed characteristics, we demonstrate that our spatio-temporal-optimized Graph RAG achieves an F1 score of $0.598$ compared to $0.007$ for traditional vector-based RAG—a nearly \textbf{85-fold improvement}. Notably, vector RAG performance degrades by $92\%$ when scaling from 100 to 2,000 documents, while our method maintains stable accuracy. We further benchmarked performance across a five-level query complexity hierarchy, finding that temporal reasoning queries saw the highest improvement ($\sim$130$\times$) and highlighting the critical role of graph construction and query parsing for complex spatio-temporal tasks.

These dual contributions—an open-source processing pipeline achieving \hl{$76.3\%$} geolocation accuracy and a retrieval architecture that maintains performance at scale—establish a replicable framework for historical document analysis in urban planning research. The methods generalize to legal document review, housing policy analysis, and other domains requiring structured, scalable reasoning over large archival collections.

%% ============================================
%% NOTES:
%% - STARA reference: footnote style (per template)
%% - GitHub: integrated naturally as contribution
%% - Requires \usepackage{url} or \usepackage{hyperref} in preamble
%% - CONFIRM: Is 76.3% the correct number? (marked with \hl{})
%% 
%% WORD COUNT: ~400 words ✓
%% ============================================
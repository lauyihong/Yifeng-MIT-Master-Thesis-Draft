% Chapter 2: Related Work
% Master's Thesis - Yifeng
% Draft Version 1.0 - FRAMEWORK ONLY

\chapter{Related Work}\label{ch:related-work}

This chapter reviews prior work across four areas relevant to our research: retrieval-augmented generation systems, graph-based retrieval methods, multi-hop and temporal reasoning benchmarks, and legal document analysis.

% ============================================
% 2.1 RAG Evolution
% ============================================
\section{Retrieval-Augmented Generation}\label{sec:rag-evolution}

\hl{[Evolution of RAG from Lewis et al. 2020 to current, most recent work from Gengcheng Mai(Spatial RAG)]}

\subsection{Vector-Based Retrieval}

\hl{[Dense retrieval, embedding models, limitations for structured reasoning]}

\subsection{Hybrid Retrieval Approaches}

\hl{[Combining sparse and dense, BM25 + embeddings]}

\subsection{Limitations for Complex Reasoning}

\hl{[Why vector similarity fails for multi-hop, cite empirical studies]}


% ============================================
% 2.2 Graph RAG Methods  
% ============================================
\section{Graph-Enhanced Retrieval}\label{sec:graph-rag-methods}

\hl{[Intro paragraph on knowledge graph integration with LLMs]}

\subsection{Microsoft GraphRAG}

\hl{[Edge et al. 2024, community detection, hierachical summarization, 70-80\% win rate on global queries]}

\subsection{LightRAG}

Guo et al. EMNLP 2025, 6x cheaper, dual-level retrieval, 80\%+ on legal docs

\subsection{HippoRAG}

Gutierrez et al. NeurIPS 2024, neurobiological inspiration, 20\% improvement on multi-hop, 10-30x cheaper than iterative

\subsection{Other Approaches}

SubgraphRAG ICLR 2025, Neo4j GraphRAG, comparison table


% ============================================
% 2.3 Multi-Hop and Temporal QA
% ============================================
\section{Multi-Hop Reasoning and Temporal Question Answering}\label{sec:multihop-temporal}

\hl{[Intro on why multi-hop is hard, temporal adds another dimension]}

\subsection{Multi-Hop QA Benchmarks}

\hl{[HotpotQA EMNLP 2018, MuSiQue TACL 2022, MultiHop-RAG COLM 2024]}

\subsection{Temporal Knowledge Graph QA}

\hl{[CronKGQA ACL 2021, TempoQR AAAI 2022, EXAQT CIKM 2021, TimelineKGQA]}

\subsection{Spatio-Temporal Reasoning}

\hl{[Limited prior work, SSTKG, GeoQA, gap our work addresses]}


% ============================================
% 2.4 Legal/Historical Document Analysis
% ============================================
\section{Legal and Historical Document Analysis}\label{sec:legal-docs}

\hl{[NLP for legal domain, historical document digitization challenges]}

\subsection{Legal NLP Systems}

\hl{[Legal-BERT, LexGLUE benchmark, contract analysis]}

\subsection{Historical Document Processing}

\hl{[OCR challenges, archaic language, Stanford STARA project]}

\subsection{Racial Covenant Research}

\hl{[Mapping Prejudice project, Segregated Seattle, prior manual efforts, gap in automated analysis]}


% ============================================
% 2.5 Summary and Positioning
% ============================================
\section{Summary and Research Gap}

\hl{[Position our work: first systematic Graph RAG evaluation on spatio-temporal legal documents, combining extraction pipeline with reasoning framework]}